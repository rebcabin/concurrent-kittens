% Created 2023-03-21 Tue 12:27
% Intended LaTeX compiler: pdflatex
\documentclass[10pt,oneside,x11names]{article}
\usepackage[utf8]{inputenc}
\usepackage[T1]{fontenc}
\usepackage{graphicx}
\usepackage{longtable}
\usepackage{wrapfig}
\usepackage{rotating}
\usepackage[normalem]{ulem}
\usepackage{amsmath}
\usepackage{amssymb}
\usepackage{capt-of}
\usepackage{hyperref}
\usepackage{minted}
\usepackage{bm}
\usepackage[T1]{fontenc}
\usepackage{cmll}
\usepackage{amsmath}
\usepackage{amssymb}
\usepackage{interval}  % must install texlive-full
\usepackage{mathtools}
\usepackage{interval}  % must install texlive-full
\usepackage[shortcuts]{extdash}
\usepackage{tikz}
\usepackage[utf8]{inputenc}
\usepackage[top=1.25in,bottom=1.25in,left=1.25in,right=1.25in,includefoot]{geometry}
\usepackage{palatino}
\usepackage{siunitx}
\usepackage{braket}
\usepackage[euler-digits,euler-hat-accent]{eulervm}
\usepackage{fancyhdr}
\pagestyle{fancyplain}
\lhead{}
\chead{\textbf{(c) Brian Beckman, 2023; Creative Commons Attribution-ShareAlike CC-BY-SA}}
\rhead{}
\lfoot{(c) Brian Beckman, 2023; CC-BY-SA}
\cfoot{\thepage}
\rfoot{}
\usepackage{lineno}
\usepackage{minted}
\usepackage{listings}
\usepackage{parskip}
\setlength{\parindent}{15pt}
\usepackage{listings}
\usepackage{xcolor}
\usepackage{textcomp}
\usepackage[atend]{bookmark}
\usepackage{mdframed}
\usepackage[utf8]{inputenc} % usually not needed (loaded by default)
\usepackage[T1]{fontenc}
\newcommand\definedas{\stackrel{\text{\tiny def}}{=}}
\newcommand\belex{BELEX}
\newcommand\bleir{BLEIR}
\newcommand\llb{low-level \belex}
\newcommand\hlb{high-level \belex}
\BeforeBeginEnvironment{minted}{\begin{mdframed}}
\AfterEndEnvironment{minted}{\end{mdframed}}
\bookmarksetup{open, openlevel=2, numbered}
\DeclareUnicodeCharacter{03BB}{$\lambda$}
\author{Brian Beckman, Ondřej Čertik}
\date{20 Mar 2023}
\title{Programming Patterns for ASR}
\hypersetup{
 pdfauthor={Brian Beckman, Ondřej Čertik},
 pdftitle={Programming Patterns for ASR},
 pdfkeywords={},
 pdfsubject={},
 pdfcreator={Emacs 26.2 of 2019-04-12, org version: 9.2.2},
 pdflang={English}}
\begin{document}

\maketitle
\setcounter{tocdepth}{2}
\tableofcontents

\setlength\parindent{0pt}

\section{Change Log}
\label{sec:org91c10ca}

2023-21-Mar :: Add section on dynamic binding.

\section{What?}
\label{sec:orgd55e568}

Let's simplify some C++ code. We can't easily run it in this
document, because it depends on a big tree of includes, but we can
look at it and figure out what it's trying to do.

The first thing to do is to indent this properly, to make the
structure obvious and to expose patterns. Standard C++ indentation
is less than helpful.

In Listing \ref{fig:typical-node-creation}, we see pushing back
(appending) an assignment statement to the body of something we're
building up. The assignment statement has five parameters:

\begin{enumerate}
\item an \emph{allocator}: \mintinline{c++}{Allocator & al}

\item a \emph{location}, e.g., \texttt{file}, \texttt{line}, \texttt{column}: \mintinline{c++}{const Location & a_loc}

\item a \emph{target} expression: \mintinline{c++}{expr_t * a_target},
usually a \texttt{Var}

\item a source or \emph{value} expression: \mintinline{c++}{expr_t * a_value}

\item an \emph{overloaded}, \mintinline{c++}{stmt_t * a_overloaded},
meaning obscure at present
\end{enumerate}

This particular assignment statement assigns or \emph{binds} the result
of a function call to a variable.

\begin{listing}[htbp]
\begin{minted}[]{c++}
body.push_back(
    al, // Allocator  & al
    ASRUtils::STMT(
        ASR::make_Assignment_t(
            al,     // Allocator      & al
            loc,    // const Location & a_loc
            // expr_t * a_target
            ASRUtils::EXPR(
                ASR::make_Var_t(
                    al,          // Allocator      & al
                    loc,         // const Location & a_loc
                    return_var)),// symbol_t       * a_v
            // expr_t * a_value
            ASRUtils::EXPR(
                ASR::make_FunctionCall_t(
                    al,          // Allocator      & al
                    loc,         // const Location & a_loc
                    s,           // symbol_t       * a_name
                    s,           // symbol_t       * a_original_name
                    call_args.p, // call_arg_t     * a_args
                    call_args.n, // size_t           n_args
                    arg_type,    // ttype_t        * a_type
                    nullptr,     // expr_t         * a_value
                    nullptr)),   // expr_t         * a_dt
            nullptr // stmt_t * a_overloaded
            )));

\end{minted}
\caption{\label{fig:typical-node-creation}Typical Node Creation}
\end{listing}

\newpage
The \texttt{STMT} and \texttt{EXPR} parameters and arguments correspond to
\emph{terms} in the ASR grammar (see the file \texttt{asr.asdl}):

\vskip 0.26cm
\begin{verbatim}
expr
  = ...
  | FunctionCall(symbol name, symbol? original_name,
          call_arg* args, ttype type, expr? value, expr? dt)
  | ...
  | Var(symbol v)
  | ...
\end{verbatim}

\vskip 0.26cm
\begin{verbatim}
stmt
  = ...
  | Assignment(expr target, expr value, stmt? overloaded)
  | ...
\end{verbatim}

Note that the ``factory functions''
\begin{itemize}
\item \mintinline{c++}{ASR::make_FunctionCall_t}
\item \mintinline{c++}{ASR::make_Var_t}
\item \mintinline{c++}{ASR::make_Assignment_t}
\end{itemize}
are automatically
generated from the ASR grammar.

Our job is to simplify the \emph{use cases} for these functions -- the
way they're called in, for example, Listing
\ref{fig:typical-node-creation}, whilst preserving automatic
generation from the grammar.

We'd like to see something like Listing
\ref{fig:simplified-usage}, which removes two levels of
indentation, removes the ``always'' parameters \texttt{al} and \texttt{loc}, and
defaults several arguments of pointer type to \mintinline{c++}{nullptr}. This is obviously easier on the eyes.

\begin{listing}[htbp]
\begin{minted}[]{c++}
body.push_back_ALLOC(
    ASRUtils::make_Assignment_STMT(
        ASRUtils::make_Var_EXPR(
            return_var), // symbol_t       * a_v
        // expr_t * a_value
        ASRUtils::make_FunctionCall_EXPR(
            s,           // symbol_t       * a_name
            s,           // symbol_t       * a_original_name
            call_args.p, // call_arg_t     * a_args
            call_args.n, // size_t           n_args
            arg_type))); // ttype_t        * a_type
\end{minted}
\caption{\label{fig:simplified-usage}Simplified Node Creation}
\end{listing}

In these figures, I supplied the indentation and commentary with
types and names of parameters by hand. Such indentation and
commentary is very helpful and straightforward to implement in the
ASR processor.

How do we get from Figure \ref{fig:typical-node-creation} to
Figure \ref{fig:simplified-usage}? There are several patterns to
exploit to reduce the size and noisiness of this code:

\begin{enumerate}
\item Every level, including the \(0^{th}\) level, \texttt{push\_back}, has an
allocator ref, \texttt{\& al}. This is probably either always the same
or it can be supplied as a stack-disciplined, dynamically
bound free variable.

\item All levels below the first have location arguments. These will
be likely different for each sub-term -- each \texttt{STMT} and
\texttt{EXPR}. These should certainly be supplied as a
stack-disciplined, dynamically bound free variable.

\item The immediate construction of \texttt{EXPR}'s around the
\texttt{FunctionCall} node and \texttt{Var} nodes to \texttt{EXPR} is pure noise.
These can be replaced by higher-level calls. Likewise with
the immediate construction of a \texttt{STMT} from the \texttt{Assignment}
node.

\item There are multiple ways to default arguments in C++. In our
case, because the defaulted arguments are last in the parameter
lists, \emph{overloads} might be the easiest way to specify them.
See this reference for more.\footnote{\url{https://en.cppreference.com/w/cpp/language/default\_arguments}}
\end{enumerate}

\section{Dynamically Bound Variables}
\label{sec:org36a4936}

\subsection{Review of Lexical Binding}
\label{sec:org131bb36}

\emph{Lexical binding} is the norm. Roughly, it means ``the place where
a variable acquires its value is obvious from just looking at the
source code.'' It's a simple concept, but tied to a bunch of
unfortunate, but necessary, terminology.

Consider a function \texttt{f} that multiplies its argument, the \emph{bound
variable} \texttt{x}, by the \emph{free variable} \texttt{y}, and returns the
product. Let \texttt{y} be a global variable for the moment; we'll modify
that later. The function \texttt{f} is in \emph{the scope}\footnote{\url{https://en.wikipedia.org/wiki/Scope\_(computer\_science)}}
of the global variable \texttt{y}.

\vskip 0.25cm
\begin{minted}[]{c++}
int y = 6;

int f (int x) { int result = x * y; return result; }

int main () { printf ("yfree = %d, f(xbound=%d) ~~> %d\n",
                      y, 7, f(7)); }
\end{minted}

\begin{verbatim}
yfree = 6, f(xbound=7) ~~> 42
\end{verbatim}


\begin{itemize}
\item Summary of confusing terminology:

\begin{itemize}
\item \texttt{y} is free in the body of \texttt{f}, the opposite of \emph{bound}.

\item \texttt{y} is \emph{not} bound in the body of \texttt{f}.

\item \texttt{y} is \emph{lexically bound} in the body of \texttt{f} and in the
bodies of any other functions in the \emph{scope} of \texttt{y}.

\item \texttt{y} is \emph{globally bound}, meaning its scope is at least the
entire file below the lexical position at which \texttt{y} acquires a
value. Its scope can be enlarged at link time with the help of
\texttt{extern} declarations in other files.
\end{itemize}

\item \texttt{x} more precisely, \texttt{x} is a \emph{parameter} of the function \texttt{f}.
When speaking imprecisely of \texttt{x} as an \emph{argument}, we mean

\begin{quote}
\emph{the  current value of the variable \texttt{x} in a particular invocation of \texttt{f}.}
\end{quote}

In the example above, it's clear that the argument \texttt{7}
is the value of the parameter \texttt{x} in the invocation \texttt{f(7)}.

\begin{itemize}
\item It's best to be very careful to distinguish parameters, which
are variables, from arguments, which are values.

\item The process by which variables acquire values is called
\textbf{\emph{binding}} the variables. In the example above, all binding
of \texttt{x} and \texttt{y} is lexical binding: it's obvious from reading
the source code wherefrom \texttt{x} and \texttt{y} acquire their values.
\end{itemize}

\item \texttt{x} is called a \textbf{\emph{bound variable}} in the body of the function
\texttt{f} because \texttt{x} is a parameter in the parameter list of \texttt{f}.
That's the only thing that \emph{bound variable}, as a single phrase,
means: \emph{in the parameter list}.

\begin{itemize}
\item One must answer ``in the parameter list of \emph{what function}?''
when speaking of a bound variable.

\item The term ``bound variable'' actually has another meaning in
another context: it can mean that the variable has a value,
without pertaining to any other properties of the variable.
This is very bad usage of terminology, because it makes the
term ``bound variable'' ambiguous without the context. We shall
be very careful to avoid that ambiguity.
\end{itemize}

\item \texttt{y} is called a \textbf{\emph{free variable}} in the body of the function
\texttt{f} because \texttt{y} is not a parameter in the parameter list of \texttt{f}.
That's the only thing that \emph{free variable} means: \emph{not in the
parameter list}. Thus, a variable may be free in the body of one
function and bound in the body of another function.

\begin{itemize}
\item Technically, \texttt{y} in some free occurrence and \texttt{y} in some bound
occurrence are \emph{not} the same variables, but the same \emph{names}
referring to different variables. This is a common source of
confusion. One must take care to distinguish \emph{names of
variables}, which are symbols, or sometimes strings, from
\emph{names of storage locations}, which are the variables
themselves. Yes, one may have the same name for different
variables. The situation becomes more piquant with pointers
and with C++ \texttt{\&} references, which permit multiple names for
the same variable.

\item One must answer ``not in the parameter list of \emph{what
function}?'' when speaking of a free variable.

\item The adjectives \emph{bound} and \emph{free}, without other
qualifications, mean \emph{``bound variable'' in the body of some
function} and \emph{``free variable'' in the body of some function},
with the terms ``bound variable'' and ``free variable'' taken as
whole phrases.
\end{itemize}

\item \texttt{y} is \emph{also} a \textbf{\emph{global variable}} in that little program
above. That means that its value, when \texttt{y} occurs free in the
body of the function \texttt{f} or in the body of any other function,
is looked up in the global environment.

\item \texttt{y} is called \textbf{\emph{lexically bound}} in the body of \texttt{f}. That means
that a human reading the program can look around the source code
and find the places where \texttt{y} acquired a value. In this case,
\texttt{y} is statically assigned the value \texttt{6} in the global
environment.

\begin{itemize}
\item Again, the terminology is confusing, and we avoid ambiguity by
never taking shortcuts with the language. \emph{Bound}, by itself,
means one thing, the opposite of \emph{free}, and \emph{lexically bound}
is not just the adverb \emph{lexically} modifying the adjective
\emph{bound}, but a whole phrase denoting an entirely separate
concept.
\end{itemize}
\end{itemize}

\subsection{Dynamically Binding \texttt{y}}
\label{sec:org0521e28}

\vskip 0.26cm
\begin{minted}[]{c++}
int y = 6;

static int push_y = y; // generalize this to a run-time stack

void bind_y(int value) { push_y = y; y = value; }

void unbind_y() { y = push_y; }

int f (int x) { int result = x * y; return result; }

int main () {
    bind_y(17);  // in Python, this would start a "with" block.
    printf ("yfree = %d, f(xbound=%d) ~~> %d\n",
                      y, 7, f(7));
    unbind_y();  // this would end the "with" block.
    printf ("but look, yfree is still %d!\n", y); }
\end{minted}

\begin{verbatim}
yfree = 17, f(xbound=7) ~~> 119
but look, yfree is still 6!
\end{verbatim}


Dynamic binding effects a temporary change to the value of a free
variable, global or not, at run time. The value of a variable is
no longer lexically obvious: one must look up a run-time stack and
not just a lexical nest of bindings. In the example above, we have
a pair of functions that work only for the variable \texttt{y}, and only
for one level of dynamic binding, but that illustrate the concept.

\begin{itemize}
\item The scope of the dynamic binding of \texttt{y} is all the code between
the call of \texttt{bind\_y} and the call of \texttt{unbind\_y}.

\item In the scope of the dynamic binding of \texttt{y}, all free occurrences
of \texttt{y} in the bodies of all functions, no matter how deeply
nested, will have the value \texttt{17}, in this instance.

\item To support nesting of binding scopes, replace the
implementations of \texttt{bind\_y} and \texttt{unbind\_y} with implementations
that employ a stack\footnote{\url{https://cplusplus.com/reference/stack/stack/}}.

\item Generalizing dynamic binding to any variable requires either
special syntax and compiler help in the programming language, as
with \texttt{binding} in
Clojure\footnote{\url{https://clojuredocs.org/clojure.core/binding}} or special
variables in Common Lisp\footnote{\url{https://wiki.c2.com/?SpecialVariable}},
or a global dictionary between variable names as strings, e.g.,
\texttt{"x"}, and their variables, say \texttt{x}, along with a stack for each
variable.

\begin{itemize}
\item Both the footnoted articles above are worth your time to read
if you're not familiar with dynamic binding.
\end{itemize}
\end{itemize}
\end{document}