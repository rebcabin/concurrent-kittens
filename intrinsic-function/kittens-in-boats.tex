% Created 2023-03-31 Fri 06:33
% Intended LaTeX compiler: pdflatex
\documentclass[10pt,oneside,x11names]{article}
\usepackage[utf8]{inputenc}
\usepackage[T1]{fontenc}
\usepackage{graphicx}
\usepackage{longtable}
\usepackage{wrapfig}
\usepackage{rotating}
\usepackage[normalem]{ulem}
\usepackage{amsmath}
\usepackage{amssymb}
\usepackage{capt-of}
\usepackage{hyperref}
\usepackage{minted}
\usepackage{bm}
\usepackage[T1]{fontenc}
\usepackage{cmll}
\usepackage{amsmath}
\usepackage{amsthm}
\usepackage{amssymb}
\usepackage{interval}  % must install texlive-full
\usepackage{mathtools}
\usepackage{interval}  % must install texlive-full
\usepackage[shortcuts]{extdash}
\usepackage{tikz}
\usepackage[utf8]{inputenc}
\usepackage[top=1.25in,bottom=1.25in,left=1.75in,right=1.75in,includefoot]{geometry}
\usepackage{palatino}
\usepackage{siunitx}
\usepackage{braket}
\usepackage[euler-digits,euler-hat-accent]{eulervm}
\usepackage{fancyhdr}
\pagestyle{fancyplain}
\lhead{}
\chead{\textbf{(c) Brian Beckman, 2023; Creative Commons Attribution-ShareAlike CC-BY-SA}}
\rhead{}
\lfoot{(c) Brian Beckman, 2023; CC-BY-SA}
\cfoot{\thepage}
\rfoot{}
\usepackage{lineno}
\usepackage{minted}
\usepackage{listings}
\usepackage{tipa}
\usepackage{parskip}
\setlength{\parindent}{15pt}
\usepackage{listings}
\usepackage{xcolor}
\usepackage{textcomp}
\usepackage[atend]{bookmark}
\usepackage{mdframed}
\usepackage[utf8]{inputenc} % usually not needed (loaded by default)
\usepackage[T1]{fontenc}
\newcommand\definedas{\stackrel{\text{\tiny def}}{=}}
\newcommand\belex{BELEX}
\newcommand\bleir{BLEIR}
\newcommand\llb{low-level \belex}
\newcommand\hlb{high-level \belex}
\newcommand{\Coloneqq}{\mathrel{\vcenter{\hbox{$:\,:\,=$}}{}}}
\newcommand\napping    [0]{0}
\newcommand\chatting   [3]{\bar{#1}\langle{#2}\rangle{}.\,#3}
\newcommand\listening  [3]{#1(#2).\,#3}
\newcommand\two        [2]{#1\mathrel{\vert}{#2}}
\newcommand\whispering [2]{(\nu\,#1)\,{#2}}
\newcommand\mama       [1]{!\,{#1}}
\newcommand{\kitOne}{\chatting{x}{z}{\napping}}
\newcommand{\kitTwo}{\listening{x}{y}{\chatting{y}{x}{\listening{x}{y}{\napping}}}}
\newcommand{\kitThree}{\listening{z}{v}{\chatting{v}{v}{\napping}}}
\newcommand{\wkit}{\whispering{x}{P}}
\newcommand{\kitTwoSuffix}{\chatting{y}{x}{\cdots}}
\newcommand{\kitTwoSuffixx}{\chatting{y}{x}{\listening{x}{y}{\napping}}}
\newcommand{\kitTwoHearSuffix}{\listening{x}{y}{\napping}}
\theoremstyle{definition}
\newtheorem{definition}{Definition}
\theoremstyle{warning}
\newtheorem{warning}{Warning}
\newcommand\say [3]{\left(\texttt{say}\thickspace{}#1\thickspace{}#2\thickspace{}#3\right)}
\newcommand\hear[3]{\left(\texttt{hear}\thickspace{}#1\thickspace{}#2\thickspace{}#3\right)}
\newcommand\kitA{\say{x}{z}{\napping}}
\newcommand\kitB{\hear{x}{y}{\say{y}{x}{\hear{x}{y}{\napping}}}}
\newcommand\kitC{\hear{z}{v}{\say{v}{v}{\napping}}}
\newcommand\knapping    [0]{\texttt{(nap)}}
\newcommand\kpar        [2]{\texttt{(par}\thickspace{}#1\thickspace{}#2\texttt{)}}
\newcommand\kwhispering [2]{\texttt{(channel}\thickspace{}#1\thickspace{}#2\texttt{)}}
\newcommand\kmama       [1]{\texttt{(repeat}\thickspace{}#1\texttt{)}}
\BeforeBeginEnvironment{minted}{\begin{mdframed}}
\AfterEndEnvironment{minted}{\end{mdframed}}
\bookmarksetup{open, openlevel=2, numbered}
\DeclareUnicodeCharacter{03BB}{$\lambda$}
\author{Brian Beckman}
\date{29 Mar 2023}
\title{Communicating Concurrent Kittens}
\hypersetup{
 pdfauthor={Brian Beckman},
 pdftitle={Communicating Concurrent Kittens},
 pdfkeywords={},
 pdfsubject={},
 pdfcreator={Emacs 26.2 of 2019-04-12, org version: 9.2.2},
 pdflang={English}}
\begin{document}

\maketitle
\setcounter{tocdepth}{2}
\tableofcontents

\setlength\parindent{0pt}

\clearpage
\section{Prologue}
\label{sec:org2d1d41a}

I must confess a mild mental disability: when I see
mathematics or programs, I see drowning kittens, and I want to
save them.

\texttt{x | y \& z}, five little kittens, \texttt{x}, \texttt{|}, \texttt{y}, \texttt{\&}, \texttt{z}
drowning in the Syntactical Sea of Forgotten Precedence.
Venus-the-boatwright rises on her half-shell, fists full of
bows and sterns. She reaches down and builds little boats,
each from a bow and a stern. She puts some little kittens in
each boat. Sometimes, she puts another little boat in a boat
in place of a kitten.

She kisses the first kitten (or little boat) in each little
boat and says ``You're the captain; take care of the others!''
She blesses the little boats and sinks back into the sea,
leaving \texttt{(\& z (| x y))} sailing safely on the waves. Venus
knows you didn't mean \texttt{(| x (\& y z))}.

Have you been burnt by C code like \texttt{x|y\&z} or \texttt{x/y*z}?
Mathematicians and Physicists normally read \(\,x/yz\,\) as
\(\,x/(yz)\,\),\footnote{Section E.2.e of the Style Guide of the
American Physical society states that multiplication has
higher precedence than division, thus implies my
statement.}\textsuperscript{,}\,\footnote{\url{https://cdn.journals.aps.org/files/styleguide-pr.pdf}}\textsuperscript{,}\,\footnote{I
have not found recent corroboration from the American
Mathematical Society, but I know I have read it in the past!}
\footnote{\url{https://math.stackexchange.com/questions/213406/does-x-yz-mean-x-yz-or-x-yz}}
but I know of no programming language that treats \texttt{x/y*z} like
that. Rather, programming languages treat it like \((x
z)/y\).\footnote{even in Mathematica, where \texttt{x / y z === x z / y}}

Perhaps you can dream, like me, of saving the drowning
kittens. Wright the boats in drydock\footnote{compiled
ahead-of-time (AOT)} or asea,\footnote{interpreted or compiled
just-in-time (JIT)} right the boats so they float,\footnote{type-check and optimize} and write \texttt{(\& z (| x y))} and
\texttt{(* z (/ x y))}! Or did you mean \texttt{(| x (\& y z))} and
\texttt{(/ x (* y z))}? Only you can know!

If you don't like bows and sterns and little boats, perhaps you can
tolerate me. My affliction is not my fault! It is like
misophonia:\footnote{\url{https://www.webmd.com/mental-health/what-is-misophonia}}
built-in, incurable, misery-making for others, inexplicable except to fellow
sufferers.\footnote{Plus, I don't like writing parsers: it's boring.}

\subsection{``What are you doing?''}
\label{sec:org43c4f0f}

Let's implement some fundamentals of the rho calculus\footnote{Meredith, L. G.; Radestock, Mattias (22 December 2005). ``A Reflective Higher-Order Calculus''. Electronic Notes in Theoretical Computer Science. 141 (5): 49–67. \url{https://doi.org/10.1016/j.entcs.2005.05.016}.}
and the pi calculus\footnote{\url{https://en.wikipedia.org/wiki/\%CE\%A0-calculus}\label{orgf4f0698}}
in a DSL\footnote{\url{https://en.wikipedia.org/wiki/Domain-specific\_language}}
made of little boats, meaning ``embedded in Clojure.'' Asynchronous
behavior is easy to model in Clojure, and there is
\texttt{clojure.spec}\footnote{\url{https://clojure.org/guides/spec}} for type
checking. Plus Clojure \emph{looks like} little boats to save
the drowning kittens. It's important for programs to look good.
Mathematics and programming \emph{are visual arts}.

\subsection{``Why are you doing this?''}
\label{sec:org252d2ad}

This is the first step in a journey toward compilers that detect
race conditions, deadlocks, starvation, and other concurrency ills
at compile time, in a type checker!

\subsection{``Why not RhoLang?''}
\label{sec:orgd5832b0}

You call up your friend Nancy and invite her to a party at your
house. She asks ``Will Ted be there?'' You think to yourself
\emph{what fun!} and innocently say, ``Yes, I think so! I invited him!''
Nancy says ``I'm terribly sorry, but I won't be able to make it to
your party. Please have a great time and give my best to
everyone!'' Nancy hangs up and thinks to herself \emph{except to Ted}.

Ted's last name is Scala.

Let's compromise. Nancy won't ever see Scala again, plus she can't
stand to see kittens drowning, i.e., implicit precedence rules.
Otherwise she likes RhoLang. Let's create something with the same
semantics as RhoLang embedded in Clojure, and call it CrowLang.

Perhaps CrowLang can inter-operate with RhoLang. They're both on
the Java Virtual Machine, after all. Or perhaps we'll write new
lcompilers\footnote{\url{https://github.com/lcompilers}\label{org223cd52}} [\textit{sic}]
for CrowLang and RhoLang. You can write an lcompiler for Scala, Nancy
doesn't care. Lcompilers are fast, flexible, modularized, and
easy to write. In fact, some day, lcompilers will use rho and pi
for internal type-checking and other formalisms!

\subsection{``Why not something better?''}
\label{sec:orgd597c3a}

Considered and rejected:

\begin{description}
\item[{Agda}] --- too obscure

\item[{Haskell or Mathematica}] --- I think Mathematica is my all-time
favorite programming language. Sadly, no one else will use it.
Haskell is in my top-five favorites, rejected for the same reason.

\item[{Python}] --- well, ok, umm, never mind, no

\item[{Common Lisp}] --- not modern, otherwise fantastic!

\item[{Racket}] --- designed for DSLs after all, but no one else uses it

\item[{Coq}] --- full of rabbit holes, otherwise lovely!

\item[{C++}] --- I would drive us both crazy.
\end{description}

There \emph{is} a path of least resistance for me, considering all
things.

\subsection{``Clojure doesn't have types!''}
\label{sec:org5f23e2e}

Not so. Clojure.spec\footnote{\url{https://clojure.org/guides/spec}} is at
least as strong as types. It's not static, that's true, but
\texttt{s/conform} \emph{could} be static.

Static or not hardly matters in our case. One can build little
boats in drydock before setting sail, or Venus-the-boatwright will
build them at sea for us. One can check types, do rewriting,
prove theorems. It's good enough.

Also, Clojure is already in our toolchain. We use it for abstract
interpretation and test generation\footnote{\url{https://github.com/rebcabin/asr-tester}}
for lfortran and lpython.\textsuperscript{\ref{org223cd52}}

\subsection{``How did you write this?''}
\label{sec:org253c443}

This is an executable document. When I produce a PDF from it, all
code blocks are executed and results reported. I like this better
than Jupyter notebooks for many reasons. This document
is an instance of Knuth's literate programming\footnote{\url{https://en.wikipedia.org/wiki/Literate\_programming}}
in org-babel.\footnote{\url{https://orgmode.org/worg/org-contrib/babel/}}

\newpage
\section{\(\Pi\) Communicating Concurrent Kittens}
\label{sec:org28ef35b}

From the wiki page:\textsuperscript{\ref{orgf4f0698}}

\begin{equation}
\begin{array}{clll}
  P, Q & \Coloneqq \\
       & | \quad \napping            & \textrm{napping kitten}    & \textrm{Do nothing; halt.}                                    \\
       & | \quad \listening{x}{y}{P} & \textrm{listening kitten}  & \textrm{Listen on channel }x\textrm{ for channel }y\textrm{.} \\
       & | \quad \chatting{x}{y}{P}  & \textrm{chatting kitten}   & \textrm{Say "}y\textrm{" on channel }x\textrm{; don't wait.}  \\
       & | \quad \two{P}{Q}          & \textrm{two kittens}       & \textrm{Run }P\textrm{ and }Q\textrm{ in parallel.}           \\
       & | \quad \whispering{x}{P}   & \textrm{whispering kitten} & \textrm{fresh channel name }x\textrm{; Use it in }P\textrm{.} \\
       & | \quad \mama{P}            & \textrm{mama cat}          & \textrm{Run copies of }P\textrm{ forever.} \\
\end{array}
\end{equation}

This pi calculus is-low-level, like \(\lambda\) calculus, only with
concurrency added. We'd have to build up numbers (like Church
numerals), Booleans, sets, functions, conditionals, loops,
everything. We'll do a little better, later. First, let's save
some drowning kittens!

\subsection{A Cartoon}
\label{sec:orgcb877c3}

Here is a tiny calculation cartoon, again from the wiki page,
showing a reduction similar to an \(\eta\)\nobreakdash-reduction in
\(\lambda\) calculus:

\begin{equation}
\begin{array}{clll}
  (\nu\,x) & (      & \kitOne \\
           & \vert  & \kitTwo \\
           & \vert  & \kitThree \\
       {}  & )
\end{array}
\end{equation}

I see four drowning kittens. Kitten Zero is a whispering
kitten \(\wkit\). She whispers ``\(x\)'' to the other kittens,
meaning ``let's talk on channel \(x\)!'' One might write:

\begin{equation}
\label{eqn:the-first-boat}
\whispering{x}{\left(
\begin{array}{clll}
 {}     & \kitOne \\
 \vert  & \kitTwo \\
 \vert  & \kitThree \\
\end{array}\right)}
\end{equation}

Kitten Zero is obviously going to be the captain of a boat
containing the remaining kittens, but we've only just started to
wright boats.

\begin{quote}
\emph{We'll say ``kitten'' and mean ``an actual kitten, or a little boat
containing kittens or more little boats.'' Each little boat contains zero-or-more kittens and zero-or-more more littler boats.}
\end{quote}

One of the three remaining kittens is chatting on channel \(x\) and
the other two are listening, one on \(x\) and the other on \(z\).
These three are doing their things two at a time,
\(\two{P}{(\two{Q}{R})}\) or \(\two{(\two{P}{Q})}{R}\), it doesn't
matter how you think about it (\emph{associativity of} par).\footnote{A
better \emph{par} boat could hold any number of kittens, in any order.
We'll get there.}

Kitten One, \(\kitOne\), chats on \(x\), ``Hey, let's continue chatting
on \(z\)!'' Then she takes a nap. Only one of the other kittens,
Kitten Two, \(\kitTwo\), listens on \(x\).\footnote{If more than one
kitten listens on the same channel, that is a classic race
condition. A compiler can warn about this directly from the syntax
of the program! At run time, only one will hear and the other will
starve, at least for a while.} She thinks, ``Oy! Here I am waiting
on \(x\) for someone to tell me where (on what channel) to continue,
and I just heard `continue on \(z\),' so I'll switch to \(z\). After I
switch, I'll have something to say, but just let me switch,
first!''

\begin{equation}
\label{eqn:the-first-chat}
\whispering{x}{\left(
\begin{array}{clll}
 {}     & \kitOne \\
 {}     & \parallel \hspace{4.5pt} \downarrow \\
 \vert  & \kitTwo \\
 {}     & \hspace{9.5pt}\downarrow\hspace{8.5pt}\downarrow \\
 \vert  & \listening{x}{z}{\chatting{z}{x}{\listening{x}{y}{\napping}}} \\
 \vert  & \kitThree \\
\end{array}\right)}
\end{equation}

The out-channel \(\bar{x}\) of Kitten One matches the in-channel \(x\)
of Kitten Two; Kitten One said something and Kitten Two heard it.
The \emph{say-prefix}, \(\bar{x}\langle{}z\rangle\), before the dot of
Kitten One, \(\kitOne\), gets gobbled up, and then Kitten One takes a
nap, \(0\). The \emph{hear-prefix}, \(x(y)\), before Kitten Two's first dot, the
first dot of \(\kitTwo\), also gets gobbled up. Plus, \(y\) changes to
\(z\) in the next say-prefix, \(\bar{y}\langle{x}\rangle\), of Kitten
Two's first suffix, \(\kitTwoSuffixx\):

\begin{equation}
\label{eqn:after-step-one}
\whispering{x}{\left(
\begin{array}{clll}
 {}     & \napping  \\
 \vert  & \chatting{z}{x}{\listening{x}{y}{\napping}} \\
 \vert  & \listening{z}{v}{\chatting{v}{v}{\napping}} \\
\end{array}\right)}
\end{equation}

This is rather like the substitution of actual arguments for
formal parameters in a function call in an ordinary programming
language.

\subsection{Free and Bound}
\label{sec:org86a324c}

Before this substitution of \(z\) for \(y\), the \(y\) in Kitten
Two's next say-prefix, \(\bar{y}\langle{x}\rangle\), is \emph{free}.
That means it must change to \(z\). However, the \(y\) in Kitten
Two's next hear-suffix, \(x(y).0\), isn't free. The \(y\) in
\(x(y)\) doesn't change to \(z\) because that \(y\) is \emph{local} to
the final suffix, \(0\). If \(0\) had more to do using \(y\), that
\(y\) would shadow the earlier \(y\). In this case, the suffix \(0\)
has no more to do; don't worry.

\emph{Bound} is a synonym for \emph{not free}. A variable \(y\) is either
free or bound in a prefix, \((\nu\,x)\),
\(\bar{y}\langle{x}\rangle\), or \(x(y)\). It can't be both.\footnote{What about the strange case \(x(x)\)? We'll solve that soon.}
Once \(y\) is bound, it's bound in all suffixes to the right up
until the next binding of \(y\). Any re-bindings of \(y\) in a
long suffix pertain to the closest binding to the left. That
closest binding must be a whispering kitten or a listening
kitten.

\newpage
\subsection{Binding}
\label{sec:org0560ca9}

There are only two ways to bind a name --- only two \emph{binding prefix forms}:

\begin{description}
\item[{whispering}] --- \(\whispering{x}{P}\) binds \(x\) in its suffixes
\(P.Q.\cdots\), \newline up until the next binding of \(x\).

\item[{listening}] --- \(\listening{x}{y}{Q}\) binds \(y\) in its suffixes
\(Q.R.\cdots\), \newline up until the next binding of \(y\).
\end{description}

\label{def:binding}
\begin{definition}{\emph{binding, scope:}}
  Each binding of a given name, say $y$, pertains to the entire
  suffix of its binding form, up until the next binding of $y$.
  That new binding
  \emph{shadows} the prior binding. This is like the
  \emph{environment model} or \emph{lexical binding} of an
  ordinary programming language.
  A sequence of binding prefixes describes a right-hugging
  nest of \emph{scopes} in which to look up values of
  bound variables.

  Shadowing, if undesirable, can be removed
  by $\alpha$\nobreakdash-renaming the new bound occurrence of $y$,
  say to $y_1$,
  bringing the prior binding of $y$ into scope of $y_1$.
  $\blacksquare$
\end{definition}

\(\alpha\)\nobreakdash-Renaming is explained immediately below in
Section \ref{renaming}.

In our example, looking at Kitten Two's suffix, \(\kitTwoSuffixx\)
in Equation \ref{eqn:the-first-boat} before any changes, one
doesn't yet know \emph{to what value} \(y\) gets bound. One only finds
out later in Equations \ref{eqn:the-first-chat} and
\ref{eqn:after-step-one}, when Kitten Two's hear-prefix, \(x(y)\),
lines up with \(\bar{x}\) in Kitten One's say-prefix,
\(\bar{x}\langle{z}\rangle\).

This usage of the word
\emph{bound} means \emph{eventually bound to something}. The term
\emph{bound} by itself can be ambiguous, because one might say
\emph{bound} either when we already do know \emph{bound to what} or when we
might only eventually know.

\subsection{Substitution}
\label{sec:orgef190d2}

Here is a general rule for \emph{substitution}, with some
terminology to be clarified:

\label{def:substitution}
\begin{definition}{\emph{substitution}:}
  When the channel $x$ of a left-most say-prefix, $\bar{x}\langle{z}\rangle$,
  equals the channel $x$ of a left-most hear-prefix, $x(y)$, the prefixes
  are gobbled up and all free occurrences of $y$ on the
  right of the hear-prefix suffer substitution of $z$ for $y$. If there
  are two or more listeners on $x$, the results are non-deterministic.
  $\blacksquare$
\end{definition}

Hiding in that little word ``non-deterministic'' is the whole value
proposition of this exercise: we're going to write programs that
can detect race conditions at compile time, in a type checker!
The compiler can warn the user. Sometimes, race conditions are
desired, but usually they're not.

\subsection{Renaming}
\label{sec:org69650a7}
\label{renaming} What if there were already some bound \(z\)'s
amongst the suffixes of free \(y\)'s? The kitten listening on
\(y\) and hearing \(z\) would have to patch that up first. It
doesn't matter what temporary name she gives to a channel, so
long as the same channel has the same bound name everywhere in
the suffixes. One might rename preexisting \(z\)'s something
like \(z_1\) so long as \(z_1\) doesn't itself collide with
preexisting names. That's \emph{alpha renaming}. It might
harmlessly un-shadow some names.

One doesn't have that problem here, but we might later.
Kittens always remember their sailorly duty to clean up messes
in their boats.

\label{def:renaming}
\begin{definition}{\emph{renaming}:}
  Prior to substitution of $z$ for a free variable $y$ in the suffixes of
  a hear-prefix,
  any bound occurrences of $z$ to the right of the hear-prefix must be
  renamed consistently lest they collide with the incoming $z$
  that replaces $y$. $\blacksquare$
\end{definition}

\subsection{Animated Cartoons}
\label{sec:org69a465f}

I can't animate cartoons in a paper, but I visualize
calculations as symbols moving around in an animated cartoon
(please forgive another of my mental afflictions:
synaesthesia). It saves me mistakes, plus the overwhelming time
spent looking for mistakes. I animate calculations
with pen and paper.

\subsection{Finishing Up}
\label{sec:orgbdfc293}

In Equation \ref{eqn:after-step-one}, Kitten Two, now
\(\chatting{z}{x}{\listening{x}{z}{\napping}}\), says on \(z\)
``Switch to \(x\), will you?'' to whomever is listening. Then she
waits and listens on \(x\) for \(y\). Kitten Three, \(\kitThree\), is
listening on \(z\) for a channel. She temporarily calls that
channel \(v\), but now she knows that \(v\) is really \(x\):

\begin{equation}
\whispering{x}{\left(
\begin{array}{clll}
 {}     & \napping  \\
 \vert  & {\listening{x}{y}{\napping}} \\
 \vert  & {\chatting{x}{x}{\napping}} \\
\end{array}\right)}
\end{equation}

See how the \(z\) chat-listen pair got gobbled up and how \(x\)
got substituted for both free \(v\)'s in Kitten Three's suffix?
If not, do an animation on paper. Kitten Three didn't have to
patch up any bound \(x\)'s, but she remembers to check. Both
occurrences of \(x\) in \(\bar{x}\langle{x}\rangle\) are free,
just as both \(v\)'s were free before substitution.

Kitten Three says ``\(x\)'' on \(x\) and takes a nap without
waiting. Kitten Two hears on \(x\) that her temporary, bound
channel name \(y\) really should be \(x\) again. She changes her
\(y\) to \(x\), notices she doesn't have any patching up or
anything else to do, and takes a nap. If you don't see it in
your mind's eye, animate it on paper.

\begin{equation}
\whispering{x}{\left(
\begin{array}{clll}
 {}     & \napping   \\
 \vert  & {\napping} \\
 \vert  & {\napping} \\
\end{array}\right)}
\end{equation}

\subsection{Something Weird Happened}
\label{sec:orgfa1992a}

In your animation, you'll see that Kitten Three becomes \(x(x)\)
after \emph{matching up} and before \emph{renaming} and \emph{substitution}. This
temporary condition appears to state that \(x\) is both bound and
free in the same prefix, and that can't be!

The resolution is that the two \(x\)'s are different \(x\)'s! The
first \(x\), outside the parentheses, is a real, free name of a
real channel --- in fact, the channel furnished by and bound
in the whispering Kitten Zero, \((\nu\,x)\). That \(x\) is subject
to \emph{matching up} with a say-prefix on \(\bar{x}\). The second
\(x\), inside the parentheses, is a bound stand-in for
\emph{whatever-x-will-become}, this time, the real
\(x\) said by  \(\bar{x}\langle{x}\rangle\). That real \(x\) gets
substituted for stand-in \(x\) in the suffix, which happens to
be \(0\), don't worry.

All the kittens are napping safely in the whisperer's boat.

\subsection{All Names are Channels}
\label{sec:org124fa05}

Every variable, \(x\), \(y\), \(z\), \(v\), stands in for a
communication channel. Sometimes one knows the channel that a
variable stands for, say a bound variable in a whisper or a
free variable before matching or after substitution. Other
times, a variable stands for a channel we'll find out about
later, say a bound variable in a hear-prefix matching. That's
all one has so far: channels, known or unknown.

Here are the stages in a reduction:

\begin{enumerate}
\item \textbf{Matching} --- a free channel \(x\) in a hear-prefix
\(x(y)\) is identical to a free channel \(\bar{x}\) in a say-prefix
\(\bar{x}\langle{z}\rangle\). Exactly one of the matching
hear-prefixes is chosen, non-deterministically. It is noted
that \(z\) will replace \(x\).

\item \textbf{Renaming} --- All bound \(z\)'s in the suffix of
\(x(y)\) are consistently renamed to prevent collisions with
the incoming \(z\).

\item \textbf{Substitution} --- All free \(y\)'s in the suffix of
\(x(y)\) are replaced with \(z\).

\item \textbf{Gobbling} --- \(x(y)\) and \(\bar{x}\langle{z}\rangle\)
are removed, exposing the first prefix of their suffixes.
\end{enumerate}

Sidestep the ``funny'' problem of \(x(x)\); never construct it.
Just gobble its predecessor hear-prefix.

\subsection{Bail the Boats!}
\label{sec:org2788cb8}

For now, we've got all kittens safely napping in the big
``whisper'' boat. But they're not \emph{dry}. They had to bail out a
\emph{lot} of water --- syntactic noise --- to keep from drowning
whilst Venus-the-boatwright was working. Venus will fix that
with some little boats \emph{inside} other boats, including the
biggest ``whisper'' boat.

Venus first bails out most of the water, leaving little skeletal,
boats-in-progress --- ordinary mathematical function notation:

\begin{equation}
\whispering{x}{\left(
\begin{array}{clll}
 {}     & \texttt{say} (x, z, \napping) \\
 \vert  & \texttt{hear}(x, y, \texttt{say}(y, x, \texttt{hear}(x, y, \napping))) \\
 \vert  & \texttt{hear}(z, v, \texttt{say}(v, v, \napping))
\end{array}\right)}
\end{equation}

There is still too much water, and some kittens
still aren't inside boats! Venus! Finish the boats:

\begin{equation}
\whispering{x}{\left(
\begin{array}{clll}
 {}     & \kitA \\
 \vert  & \kitB \\
 \vert  & \kitC \\
\end{array}\right)}
\end{equation}

Venus! You're not done! Everything must be a kitten or a boat!

\vskip 0.26cm
\begin{minted}[]{clojure}
(channel x
  (par (say x z 0)
       (par (hear x y
              (say y x
                (hear x y 0)))
            (hear z v
             (say v v 0)))))
\end{minted}

Hooray, all the kittens are safe and dry! But they can't nap, yet.
Venus! Rearrange the boats so kittens can chat and then nap!

\vskip 0.26cm
\begin{minted}[]{clojure}
(channel x
  (par (par (say  x z 0)  ;; Oooh!, x's line up!
            (hear x y
                  (say y x
                       (hear x y 0))))
       (hear z v
             (say v v 0))))
\end{minted}

This is great because there is a rule that says whenever a
\texttt{say} and a \texttt{hear} line up their channels, rename, substitute
and gobble up one \texttt{say} and its matching \texttt{hear}:

\vskip 0.26cm
\begin{minted}[]{clojure}
(channel x
  (par (par 0
            (say z x
                 (hear x y 0)))
       (hear z v
             (say v v 0))))
\end{minted}

Darn it! Venus! Rearrange the \texttt{par} boats again, (it's always OK to do
that):

\vskip 0.26cm
\begin{minted}[]{clojure}
(channel x
  (par 0
       (par (say  z x (hear x y 0))
            (hear z v (say  v v 0)))))
\end{minted}

Substitute and gobble:

\vskip 0.26cm
\begin{minted}[]{clojure}
(channel x
  (par 0
       (par (hear x y 0)
            (say  x x 0))))
\end{minted}

One more time:

\vskip 0.26cm
\begin{minted}[]{clojure}
(channel x
  (par 0 (par 0 0)))
\end{minted}

Inside a \texttt{par} boat, it doesn't matter whether you write
\texttt{hear} before \texttt{say} or \texttt{say} before \texttt{hear} --- \texttt{par} is the
captain and doesn't care; \texttt{par} is commutative. Also, because
any number of napping kittens in \texttt{par} boats is equivalent to
a all the kittens napping, write

\vskip 0.26cm
\begin{minted}[]{clojure}
(channel x 0)
\end{minted}

\newpage
Finally, because there is nothing to do with channel \(x\), The
whispering kitten can nap, too.

\vskip 0.26cm
\begin{minted}[]{clojure}
0
\end{minted}

Thanks, Venus!

\subsection{Kitten Boat Calculus}
\label{sec:orgbb76fed}

This is what Venus-the-boatwright had in mind whilst she built:

\begin{equation}
\begin{array}{clll}
  K, L & \Coloneqq \\
       & | \quad \knapping            & \textrm{napping kitten}    & \textrm{Do nothing; halt.}                                   \\
       & | \quad \hear{x}{y}{K}       & \textrm{listening kitten}  & \textrm{Listen on channel }x\textrm{ for channel }y.         \\
       & | \quad \say{x}{y}{K}        & \textrm{chatting kitten}   & \textrm{Say "}y\textrm{" on channel }x\textrm{; don't wait.} \\
       & | \quad \kpar{K}{L}          & \textrm{two kittens}       & \textrm{Run }K\textrm{ and }L\textrm{ in parallel.}          \\
       & | \quad \kwhispering{x}{K}   & \textrm{whispering kitten} & \textrm{fresh channel name }x\textrm{; use it in }K\textrm{.}\\
       & | \quad \kmama{K}            & \textrm{mama cat}          & \textrm{run copies of }K\textrm{ forever}\textrm{.}          \\
\end{array}
\end{equation}

\section{Channels and Names}
\label{sec:org290a40b}

The kittens are named Kitten One, Kitten Two, and Kitten
Three. These aren't names in kitten-speak, not names for
channels like \(x\) and \(y\). These are names in boat-speak, just
so one doesn't write out the full boats over and over again.

Let's run some real code! For technical reasons, there is some
punctuation --- dots and quote marks here and there --- for
kittens written out in Clojure.

\subsection{Kit-1}
\label{sec:org0195745}

\vskip 0.26cm
\begin{minted}[]{clojure}
(def kit-1
  (say. 'x 'z (nap.)))
\end{minted}

Notice that when \texttt{kit-1} eventually takes a nap, she's not
saying or hearing anything. \emph{The free names of \texttt{(nap)}, the
names subject to substitution, are the empty set}:

\vskip 0.26cm
\begin{minted}[]{clojure}
(free-names (nap.))
\end{minted}

\begin{verbatim}
#{}
\end{verbatim}


In fact, the names that \texttt{kit-1} will eventually know about while
napping, the \emph{bound names, subject to renaming}, are also the
empty set:

\vskip 0.26cm
\begin{minted}[]{clojure}
(bound-names (nap.))
\end{minted}

\begin{verbatim}
#{}
\end{verbatim}


Before she naps, Kitten One says \(z\) on \(x\), so both those names
are free for \texttt{kit-1}, meaning she just barks them out. They don't
stand for anything else in potential suffixes of \texttt{kit-1}:

\vskip 0.26cm
\begin{minted}[]{clojure}
(free-names kit-1)
\end{minted}

\begin{verbatim}
#{x z}
\end{verbatim}


\texttt{Kit-1} doesn't wait for any names before nap-time, so her
\emph{bound names} are the empty set:

\vskip 0.26cm
\begin{minted}[]{clojure}
(bound-names kit-1)
\end{minted}

\begin{verbatim}
#{}
\end{verbatim}

\subsection{Kit-2}
\label{sec:org7aeafef}

Kitten Two listens on \(x\) for bound \(y\), then says, on whatever
\(y\) becomes, ``\(x\)''.

\vskip 0.26cm
\begin{minted}[]{clojure}
(def kit-2
  (hear. 'x 'y
   (say. 'y 'x
    (hear. 'x 'y (nap.)))))
\end{minted}

\emph{We} know that in her immediately-next say-prefix,
\texttt{(say y x)}, \(y\) is a free variable and subject to
substitution. It eventually becomes \(z\), but \emph{she} doesn't
know so yet. She only knows that she will \emph{eventually} know
that \(y\) stands for \(z\); \(y\) is eventually bound, thus bound.

\vskip 0.26cm
\begin{minted}[]{clojure}
(bound-names kit-2)
\end{minted}

\begin{verbatim}
#{y}
\end{verbatim}


Kitten Two's final activity is to listen on \(x\) for
whatever-\(y\)-becomes. In that final activity, in isolation,
she doesn't know whether she will ever know \(x\), so the free
variables --- subject to substitution --- of that final
activity had better include \(x\).

\vskip 0.26cm
\begin{minted}[]{clojure}
(do (def kit-2-final
      (hear. 'x 'y (nap.)))
    (free-names kit-2-final))
\end{minted}

\begin{verbatim}
#{x}
\end{verbatim}


By nap-time, she'll know what \(y\) stands for, but she won't
use it while napping; \(y\) is eventually bound thus bound in
her final activity:

\vskip 0.26cm
\begin{minted}[]{clojure}
(bound-names kit-2-final)
\end{minted}

\begin{verbatim}
#{y}
\end{verbatim}


\newpage
In her next-to-last activity, which includes her last activity,
she will know what \(y\) is, so it is bound:

\vskip 0.26cm
\begin{minted}[]{clojure}
(bound-names
 (say. 'y 'x
  kit-2-final))
\end{minted}

\begin{verbatim}
#{y}
\end{verbatim}


Kit-2 never uses \(x\). She just passes \(x\) along, so it's free:

\vskip 0.26cm
\begin{minted}[]{clojure}
(free-names kit-2)
\end{minted}

\begin{verbatim}
#{x}
\end{verbatim}

\subsection{Kit-3}
\label{sec:org00037fc}

Kitten Three listens on \(z\) for \(v\) --- a temporary name --- then
says ``\(v\)'' on \(v\): after substitution of something for \(v\):

\vskip 0.26cm
\begin{minted}[]{clojure}
(def kit-3
  (hear. 'z 'v
   (say. 'v 'v (nap.))))
\end{minted}

Her bound names include \(v\), at least until it becomes free
before substitution:

\vskip 0.26cm
\begin{minted}[]{clojure}
(bound-names kit-3)
\end{minted}

\begin{verbatim}
#{v}
\end{verbatim}


Her free names --- subject to substitution --- include \(z\):

\vskip 0.26cm
\begin{minted}[]{clojure}
(free-names kit-3)
\end{minted}

\begin{verbatim}
#{z}
\end{verbatim}


Can you write down the free and bound names in her last activity,
\texttt{(say 'v 'v)}? Here are spoilers:

\vskip 0.26cm
\begin{minted}[]{clojure}
(let [kit-3-last (say. 'v 'v (nap.))]
  (println (free-names kit-3-last))
  (println (bound-names kit-3-last)))
\end{minted}

\begin{verbatim}
#{v}
#{}
\end{verbatim}

\subsection{Kitten Zero --- the Whisper Boat}
\label{sec:org46b60ad}

The bound names of Kitten Zero, captain of the Whisper Boat,
include all the bound names of the other kittens, so had better be
\(x\) for her own, \(y\) from Kitten Two, and \(v\) from kitten Three:

\vskip 0.26cm
\begin{minted}[]{clojure}
(do (def whisper-boat
      (channel. 'x
                (par. kit-1
                      (par. kit-2 kit-3))))
    (bound-names whisper-boat))
\end{minted}

\begin{verbatim}
#{x y v}
\end{verbatim}


Can you write out her free names? Here is a spoiler:

\vskip 0.26cm
\begin{minted}[]{clojure}
(free-names whisper-boat)
\end{minted}

\begin{verbatim}
#{z}
\end{verbatim}


The free names --- subject to substitution --- include only \(z\)
from both Kitten One, who barks them out in \texttt{(say. 'x 'z)}, and
Kitten Three, who listens on \(z\) for a substitution:

\section{Change Log}
\label{sec:orgb750c40}

2023-30-Mar :: weeding out the ``we's''

2023-29-Mar :: Many small corrections.

2023-28-Mar :: Done building boats.

2023-26-Mar :: Current version.

2023-22-Mar :: Start.
\end{document}